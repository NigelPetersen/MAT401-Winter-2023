\documentclass[11pt]{article}
\usepackage[english]{babel}
\usepackage[utf8]{inputenc}
\usepackage[dvipsnames]{xcolor}
\usepackage[most]{tcolorbox}
\usepackage[linguistics]{forest}
\usepackage{fancyhdr}
\usepackage{amsmath}
\usepackage{amssymb}
\usepackage{amsfonts}
\usepackage{amstext}
\usepackage{amsmath,amssymb,amsthm, thmtools}
\usepackage{tikz,lipsum,lmodern}
\usepackage{array}
\usepackage{marvosym}
\usepackage{lastpage}
\usepackage{wasysym}
\usepackage{multicol}
\usepackage{tikz-cd}
\usepackage{array}
\usepackage{dirtytalk}
\usepackage{qtree}
\usepackage{framed}
\usepackage{enumitem}
\usepackage[hyperfootnotes=false]{hyperref}
\hypersetup{
colorlinks=true,
linkcolor={mypink}}

\definecolor{mypink}{RGB}{255, 50, 147}
\definecolor{pink}{RGB}{255, 0, 147}

\setlength{\textheight}{9.3in}
\setlength{\topmargin}{-0.7in}
\setlength{\textwidth}{6.5in}
\setlength{\oddsidemargin}{0in}
\setlength{\evensidemargin}{0in}
\setlength{\parskip}{6pt}
\setlength{\parindent}{0pt}

\def\mbb{\mathbb}
\def\mb{\mathbf}
\def\mc{\mathcal}
\def\R{\mbb{R}}
\def\Q{\mbb{Q}}
\def\Z{\mbb{Z}}
\def\C{\mbb{C}}
\def\N{\mbb{N}}
\def\F{\mbb{F}}
\def\K{\mathbb{K}}
\def\T{\mc{T}}
\def\A{\mc{A}}
\def\X{\mathfrak{X}}
\def\E{\exists}
\def\e{\epsilon}
\def\d{\delta}
\def\Ra{\Rightarrow}
\def\vs{\vspace{5mm}}
\def\La{\Leftarrow}
\def\and{\quad \text{and} \quad}
\def\rad{\mathrm{rad}}
\def\tt{\texttt}
\def\Char{\mathrm{char}}

\def\nemt{\neq \emptyset}
\newcommand{\dif}[2]{\frac{d#1}{d#2}}
\newcommand{\pardif}[2]{\frac{\partial #1}{\partial #2}}
\newcommand{\twopardif}[2]{\frac{\partial ^2 #1}{\partial #2^2}}
\newcommand{\Matrix}[1]{\begin{pmatrix} #1 \end{pmatrix}}
\newcommand{\Lint}[1]{\ointctrclockwise_{#1}}
\newcommand{\Ang}[1]{\left\langle #1 \right\rangle}
\newcommand{\Inn}{\langle \cdot , \cdot \rangle}
\newcommand{\norm}[1]{\left\| #1 \right\|}
\newcommand{\blanknorm}[1]{\| \cdot \|_{#1}}
\newcommand{\epfrac}[1]{\frac{\e}{#1}}
\newcommand{\ext}[1]{\tilde{\mb #1}}

\renewcommand{\epsilon}{\varepsilon}
\renewcommand{\bar}{\overline} 
\renewcommand{\hat}{\widehat}
%\renewcommand{\qedsymbol}{\FAX}


\declaretheoremstyle[
  headfont=\color{mypink}\normalfont\bfseries,
%  bodyfont=\color{red}\normalfont\itshape,
]{pink}

\declaretheoremstyle[
  headfont=\color{black}\normalfont\bfseries,
%  bodyfont=\color{red}\normalfont\itshape,
]{boxedsolution}

\theoremstyle{pink}
\newtheorem{definition}{Definition}[section]

\theoremstyle{boxedsolution}
\newtheorem*{bproof}{Proof}
\newtheorem*{solution}{Solution}
  
\theoremstyle{definition}
\newtheorem{lemma}{Lemma}
\newtheorem{theorem}{Theorem}
\newtheorem{corollary}[definition]{Corollary}
\newtheorem{proposition}[definition]{Proposition}
\newtheorem{example}[definition]{Example}
\newtheorem{exercise}[definition]{Exercise}
\newtheorem{question}{Question}
\newtheorem{hquestion}{Hard Question}

\newtheoremstyle{claim}
{\topsep}{\topsep}{}{}%              
{\bfseries}{:}%             
{5pt plus 1pt minus 1pt}{}%             
\theoremstyle{claim}
% change this to \newtheorem*{claim}{Claim} to leave things un-numbered
\newtheorem*{claim}{Claim}



\newenvironment{boxsol}
    {\begin{framed}
    \begin{solution}
    }
    {
    \end{solution}    
    \end{framed}}
    
\newenvironment{boxproof}
    {\begin{framed}
    \begin{bproof}
    }
    {
    \end{bproof} 
    \end{framed}}

\pagestyle{fancy}
\fancyhf{}
\lhead{MAT401 - Polynomial Equations and Fields}
\rhead{Exam Practice}
\cfoot{\thepage}

\begin{document}

\thispagestyle{empty}

\begin{framed}
    \begin{center}
        \large\textbf{MAT401: Polynomial Equations and Fields --- Exam Practice} \vspace{3mm}\\
        \textbf{Topics:} Rings, Fields, Galois Theory.
    \end{center}
\end{framed}

\section*{Ring Theory}

\begin{question}
    Let $R$ be a commutative ring and define the \textit{nilradical of $R$} by 
    \[
    \mathrm{Nil}(R) = \{r \in R: r^n = 0 \text{ for some } n \in \N\}
    \]
    namely, the ideal of nilpotent elements in $R$. Prove that $\mathrm{Nil}(R/\mathrm{Nil}(R))$ is trivial.
\end{question}

\begin{question}
    Let $R$ be a finite, commutative ring with unity. Prove that every prime ideal is maximal.
\end{question}

\begin{question}
    Let $R$ be a ring with unity such that  $|R| = 401^2$. Prove that $R$ is commutative.
\end{question}

\begin{question}
    Show by example that there is a commutative ring $R$ and a maximal ideal $I \trianglelefteq R$ that is \textit{not} prime. \textit{Note:} Of course, $R$ cannot contain unity.
\end{question}

% \begin{question}
%     Let $C([0,1];\R)$, denote the collection of continuous functions $f:[0,1] \to \R$. For $\alpha \in [0,1]$, consider the evaluation map $E_{\alpha}: C([0,1];\R) \to \R$ defined by $E_{\alpha}(f) = f(\alpha)$. 
%     \begin{enumerate}[label=(\alph*)]
%         \item Prove that $\ker(E_{\alpha})$ is a maximal ideal. \textit{Hint: First isomorphism theorem}

%         \item \textbf{\textcolor{pink}{Bonus:}} Prove that \textit{every} maximal ideal of $C([0,1]; \R)$ arises as the kernel of an evaluation map, namely if $J \trianglelefteq C([0,1]; \R)$ is maximal, then there is $\alpha \in [0,1]$ for which $J = \ker(E_{\alpha})$. \textit{Note:} This requires analysis.
%     \end{enumerate}
% \end{question}

\begin{question}
If $R$ is a commutative ring for which every proper ideal is prime, must $R$ be a field?
\end{question}

\section*{Field Theory}

\setcounter{question}{0}

\begin{question}
    Let $F$ be a field for which $\Char(F) = p$ for a prime $p$, and $q(x) \in F[x]$ an irreducible polynomial of degree $n$. Prove that $K := F[x] / \Ang{q(x)}$ is a field of characteristic $p$ with $p^n$ elements. \textit{Note:} This is a way that we can construct fields of order $p^n$ for any prime $p$ and $n \in \N$.
\end{question}

\begin{question}
    Let $p$ and  $q$ be distinct primes.
    \begin{enumerate}[label=(\alph*)]
        \item Prove that $\Q(\sqrt{p}) \cong \Q(\sqrt{q})$ if and only if $p = q$. 

        \item Prove that $\Q(\sqrt{p} + \sqrt{q}) = \Q(\sqrt{p}, \sqrt{q})$.
    \end{enumerate}
\end{question}

\begin{question}
    Let $F$ be a field, $q(x) \in F[x]$ a polynomial of degree $n$, and $K$ a splitting field of $q(x)$ over $F$. Prove that $[K:F] \leq n!$
\end{question}

\begin{question}
    Let $F$ be a field, and $\alpha \in F$ algebraic over $F$ for which $[F(\alpha): F] = p$, for some prime $p$. Prove that for any $k \in \{1, \dots, p-1\}$, we have $F(\alpha^k) = F(\alpha)$. \textit{Hint:} Use the degree lemma.
\end{question}

\begin{question}
    Let $K$ be an extension of a field $F$ for which $[K:F] = 401$. Prove that there is $\alpha \in K$ for which $K = F(\alpha)$.
\end{question}

% \begin{question}
%     In this question, we will correct an incorrect proof.
%     \begin{enumerate}[label=(\alph*)]
%         \item Given an example of a field $K$ and a homomorphism $f:K \to K$ for which $f \not\equiv \mathrm{id}_K$ (i.e there is $x \in K$ for which $f(x) \neq x$.)

%         \item Let $K$ be a field and consider the following incorrect claim and proof
%         \begin{claim}
%         The only homomorphism $f: K \to K$ is $f \equiv \mathrm{id}_K$.
%         \end{claim} 
%         \begin{proof}
%         As $K$ is a one dimensional vector space over $K$ with basis $\beta = \{1\}$, it suffices to show $[f(x)]_{\beta} = [x]_{\beta}$, since the map $\phi: K \to \R$ via $x \mapsto [x]_{\beta}$ (the representation of $x$ in $\beta$ coordinates) is an isomorphism. Recall from linear algebra that for any $x \in K$, we have $[f(x)]_{\beta} = [f]_{\beta}^{\beta}[x]_{\beta}$, where $[f]_{\beta}^{\beta} = \begin{bmatrix} [f(1)]_{\beta} \end{bmatrix}$ is the matrix representation of $f$ with respect to $\beta$. As $f$ is a homomorphism of fields, we have $f(1) = 1$, and hence $[f]_{\beta}^{\beta} = [1]$ (the matrix 1). Thus, as $[x]_{\beta} = [x]$ (the matrix $x$) since $x = x\cdot 1$, it follows that $[f(x)]_{\beta} = [x]_{\beta}$, and hence $f(x) = x$.
%         \end{proof}
%         Where exactly does this "proof" break down?
%     \end{enumerate}
% \end{question}

\newpage

\section*{Galois Theory}

\setcounter{question}{0}

\begin{question}
    Let $K$ be an extension of a field $F$.
    \begin{enumerate}[label=(\alph*)]
        \item If $M$ is an intermediate extension, namely $F \subseteq M \subseteq K$, state and prove a relationship between $\mathrm{Gal}(K/M)$ and $\mathrm{Gal}(K/F)$.

        \item If $H_1 \leq H_2$ are subgroups of $\mathrm{Gal}(K/F)$, state and prove a relationship between the fixed fields of $H_1$ and $H_2$.
    \end{enumerate}
\end{question}

\begin{question}
Prove that $\operatorname{Gal}(\Q(\sqrt{3}+ \sqrt{5})/\Q) \cong \Z_2 \oplus \Z_2$. 
\end{question}

\begin{question}
Compute the Galois group of $f(x) = x^4-4x^2+2$ over $\Q$.
\end{question}

\begin{question}
Let $\alpha =  i\frac{\sqrt{3}}{2}-\frac{1}{2}$ and $\beta = \sqrt[3]{2}$. Prove that $\operatorname{Gal}(\Q(\alpha, \beta)/\Q) \cong S_3$.
\end{question}

\begin{question}
Compute the Galois group of $f(x) = x^5-2$ over $\Q$.
\end{question}

% \section*{Hard Questions \smiley}

% \begin{hquestion}
    
% \end{hquestion}

% \begin{hquestion}
    
% \end{hquestion}

% \begin{hquestion}
%     Let $q(x) \in \Q[x]$ be an irreducible polynomial containing exactly 3 real roots. Prove that $\mathrm{Gal}(p(x)) \cong S_5$. \textit{Hint:} The group $S_5$ is uniquely characterized by containing a transposition and a 5-cycle.
% \end{hquestion}



\end{document}
