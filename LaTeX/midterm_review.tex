\documentclass[11pt]{article}
\usepackage[english]{babel}
\usepackage[utf8]{inputenc}
\usepackage[dvipsnames]{xcolor}
\usepackage[most]{tcolorbox}
\usepackage[linguistics]{forest}
\usepackage{fancyhdr}
\usepackage{amsmath}
\usepackage{amssymb}
\usepackage{amsfonts}
\usepackage{amstext}
\usepackage{amsmath,amssymb,amsthm, thmtools}
\usepackage{tikz,lipsum,lmodern}
\usepackage{array}
\usepackage{marvosym}
\usepackage{lastpage}
\usepackage{wasysym}
\usepackage{multicol}
\usepackage{tikz-cd}
\usepackage{array}
\usepackage{dirtytalk}
\usepackage{qtree}
\usepackage{framed}
\usepackage{enumitem}
\usepackage[hyperfootnotes=false]{hyperref}
\hypersetup{
colorlinks=true,
linkcolor={mypink}}

\definecolor{mypink}{RGB}{255, 50, 147}
\definecolor{pink}{RGB}{255, 0, 147}

\setlength{\textheight}{9.3in}
\setlength{\topmargin}{-0.7in}
\setlength{\textwidth}{6.5in}
\setlength{\oddsidemargin}{0in}
\setlength{\evensidemargin}{0in}
\setlength{\parskip}{6pt}
\setlength{\parindent}{0pt}

\def\mbb{\mathbb}
\def\mb{\mathbf}
\def\mc{\mathcal}
\def\R{\mbb{R}}
\def\Q{\mbb{Q}}
\def\Z{\mbb{Z}}
\def\C{\mbb{C}}
\def\N{\mbb{N}}
\def\F{\mbb{F}}
\def\K{\mathbb{K}}
\def\T{\mc{T}}
\def\A{\mc{A}}
\def\X{\mathfrak{X}}
\def\E{\exists}
\def\e{\epsilon}
\def\d{\delta}
\def\Ra{\Rightarrow}
\def\vs{\vspace{5mm}}
\def\La{\Leftarrow}
\def\and{\quad \text{and} \quad}
\def\rad{\mathrm{rad}}
\def\tt{\texttt}

\def\nemt{\neq \emptyset}
\newcommand{\dif}[2]{\frac{d#1}{d#2}}
\newcommand{\pardif}[2]{\frac{\partial #1}{\partial #2}}
\newcommand{\twopardif}[2]{\frac{\partial ^2 #1}{\partial #2^2}}
\newcommand{\Matrix}[1]{\begin{pmatrix} #1 \end{pmatrix}}
\newcommand{\Lint}[1]{\ointctrclockwise_{#1}}
\newcommand{\Ang}[1]{\left\langle #1 \right\rangle}
\newcommand{\Inn}{\langle \cdot , \cdot \rangle}
\newcommand{\norm}[1]{\left\| #1 \right\|}
\newcommand{\blanknorm}[1]{\| \cdot \|_{#1}}
\newcommand{\epfrac}[1]{\frac{\e}{#1}}
\newcommand{\ext}[1]{\tilde{\mb #1}}

\renewcommand{\epsilon}{\varepsilon}
\renewcommand{\bar}{\overline} 
\renewcommand{\hat}{\widehat}
%\renewcommand{\qedsymbol}{\FAX}


\declaretheoremstyle[
  headfont=\color{mypink}\normalfont\bfseries,
%  bodyfont=\color{red}\normalfont\itshape,
]{pink}

\declaretheoremstyle[
  headfont=\color{black}\normalfont\bfseries,
%  bodyfont=\color{red}\normalfont\itshape,
]{boxedsolution}

\theoremstyle{pink}
\newtheorem{definition}{Definition}[section]

\theoremstyle{boxedsolution}
\newtheorem*{bproof}{Proof}
\newtheorem*{solution}{Solution}
  
\theoremstyle{definition}
\newtheorem{lemma}{Lemma}
\newtheorem{theorem}{Theorem}
\newtheorem{corollary}[definition]{Corollary}
\newtheorem{proposition}[definition]{Proposition}
\newtheorem{example}[definition]{Example}
\newtheorem{exercise}[definition]{Exercise}
\newtheorem{question}{Question}

\newtheoremstyle{claim}
{\topsep}{\topsep}{}{}%              
{\bfseries}{:}%             
{5pt plus 1pt minus 1pt}{}%             
\theoremstyle{claim}
% change this to \newtheorem*{claim}{Claim} to leave things un-numbered
\newtheorem{claim}{Claim}



\newenvironment{boxsol}
    {\begin{framed}
    \begin{solution}
    }
    {
    \end{solution}    
    \end{framed}}
    
\newenvironment{boxproof}
    {\begin{framed}
    \begin{bproof}
    }
    {
    \end{bproof} 
    \end{framed}}

\pagestyle{fancy}
\fancyhf{}
\lhead{MAT401 - Polynomial Equations and Fields}
\rhead{Midterm Practice}
\cfoot{Page \thepage \ of \pageref{LastPage}}

\begin{document}

\thispagestyle{empty}

\begin{framed}
    \begin{center}
        \large\textbf{MAT401: Polynomial Equations and Fields --- Midterm Practice} \vspace{3mm}\\
        \textbf{Topics:} Rings, Ideals, Quotients, Fields, Homomorphisms.
    \end{center}
\end{framed}

\vs

\begin{question}
Denote by $C(\R)$ the ring of continuous functions $f:\R \to \R$ under pointwise addition and multiplication. For points $a_1, \dots, a_n \in \R$, define 
\[
Z(a_1, \dots, a_n) = \{f \in C(\R): f(a_i) = 0 \ \forall i = 1, \dots, n\}
\]
Prove that $Z(a_1, \dots, a_n) \trianglelefteq C(\R)$. Is $Z(a_1, \dots, a_n)$ necessarily a prime ideal for any choice of points? 
\end{question}

\begin{question}
Let $R$ be an integral domain. Prove that $R[x]$, the ring of polynomials in the variable $x$ with coefficients over $R$, is also an integral domain.
\end{question}


\begin{question}
If $R$ is a ring and $K$ is a field for which there is a ring isomorphism $f: K \to R$, prove that $R$ is a field, and hence $f$ is a field isomorphism.
\end{question}

\begin{question}
Let $R$ be a ring, with $I$ and $J$ ideals of $R$. 
\begin{enumerate}
    \item[(a)] Prove the ideal test: A nonempty subset $I$ of $R$ is an ideal of $R$ if and only $x - y \in I$ and $ax\in I$ for all $x,y \in I$ and $a \in R$. (This essentially combines the subring test with the definition of an ideal)
    
    \item[(b)] Show that $I + J := \{x+y: x \in I, y \in J\}$ is an ideal of $R$.
    
    \item[(c)] Show that $IJ = \left\{ \sum_{i=1}^n x_iy_i: n \in \N, x_i \in I, y_i \in J \right\}$ is an ideal of $R$.
\end{enumerate}
\end{question}

\begin{question}
Fix a prime $p \in \N$, and recall that $\sqrt{p} \notin \Q$. Define the ideal $I_p = \Ang{x^2-p}$.
\begin{enumerate}
    \item[(a)] Prove that $\Q[x] / I_p$ is a field.
    
    \item[(b)] Prove that $\Q[x]/ I_p \cong \Q(\sqrt{p}) = \{a+b\sqrt{p}: a, b \in \Q\}$. \textit{Hint: Associate $[x]$ and $\sqrt{p}$}. 
    
    \item[(c)] \textcolor{pink}{\textbf{Bonus:}} Prove that $\Q[x]/I_p \cong \Q[x]/I_q$ if and only if $p = q$. (This is diving into material we'll see later on in the course, so don't worry about it too much now)
\end{enumerate}
\end{question}

\begin{question}
    Let $R,S$ be commutative rings with unity, and $f: R \to S$ a surjective homomorphism.
    \begin{enumerate}
        \item[(a)] Prove that for any ideals $I \trianglelefteq R$ and $J \trianglelefteq S$, we have $f(I) \trianglelefteq S$ and $f^{-1}(J) \trianglelefteq R$.
        
        \item[(b)] Suppose now that $f:R \to S$ is an isomorhpism, and consider the corresponding induced maps $F:R/I \to S/f(I)$ and $G:R/f^{-1}(J) \to S/J$ by
            \[
            F(r+I) = f(r) + f(I) \and G(r + f^{-1}(J)) = f(r) + J
            \]
        Prove that $F$ and $G$ are isomorphisms.
            
        \item[(c)] \textcolor{pink}{\textbf{Bonus:}} Suppose $I \trianglelefteq R, J \trianglelefteq S$ are prime. Are either of $f(I) \trianglelefteq S, f^{-1}(J) \trianglelefteq R$ prime? What if we replace prime with maximal? (Don't worry too much about this one, it is much easier to prove with an additional result you don't know yet)
    \end{enumerate}
\end{question}

\begin{question}
Let $R$ and $S$ be rings with unity. Prove that $R \oplus S$ cannot be a field.
\end{question}

\begin{question}
An ideal $I$ in a ring $R$ with unity is called \textit{reducible} if there are ideals $J_1, J_2 \trianglelefteq R$ for which $I = J_1 \cap J_2$ and $I \subsetneq J_1, J_2$, and is called irreducible otherwise. Prove that an ideal $I$ of $R$ is prime if and only if it is irreducible. (This furthers illustrates the idea that prime ideals behave like prime numbers)
\end{question}

\begin{question}
Let $\R[x,y]$ be the ring of polynomials in the variables $x$ and $y$ over $\R$. Prove that $\R[x,y]/\Ang{y} \cong \R[x]$. \textit{Hint: Use the first isomorphism theorem.}
\end{question}

\begin{question}
Let $R$ be a ring, and $I,J \trianglelefteq R$ for which $I \subseteq J$. Prove that $I \trianglelefteq J$ and use the map $f: R/I \to R/J$ defined by $f(x + I) = x + J$ to show that
\[
J/I \trianglelefteq R/I \and R/J \cong (R/I)/(J / I)
\]
You'll have to argue that the map is a well-defined homomorphism first, then look at the kernel and make use of a theorem.
\end{question}

\end{document}