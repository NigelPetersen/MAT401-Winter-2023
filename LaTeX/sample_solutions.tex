\documentclass[11pt]{article}
\usepackage[english]{babel}
\usepackage[utf8]{inputenc}
\usepackage[dvipsnames]{xcolor}
\usepackage[most]{tcolorbox}
\usepackage[linguistics]{forest}
\usepackage{fancyhdr}
\usepackage{amsmath}
\usepackage{amssymb}
\usepackage{amsfonts}
\usepackage{amstext}
\usepackage{amsmath,amssymb,amsthm, thmtools}
\usepackage{tikz,lipsum,lmodern}
\usepackage{array}
\usepackage{marvosym}
\usepackage{lastpage}
\usepackage{multicol}
\usepackage{tikz-cd}
\usepackage{array}
\usepackage{dirtytalk}
\usepackage{qtree}
\usepackage{framed}
\usepackage{enumitem}
\usepackage[hyperfootnotes=false]{hyperref}
\hypersetup{
colorlinks=true,
linkcolor={mypink}}

\definecolor{mypink}{RGB}{255, 50, 147}
\definecolor{pink}{RGB}{255, 0, 147}

\setlength{\textheight}{9.3in}
\setlength{\topmargin}{-0.7in}
\setlength{\textwidth}{6.5in}
\setlength{\oddsidemargin}{0in}
\setlength{\evensidemargin}{0in}
\setlength{\parskip}{6pt}
\setlength{\parindent}{0pt}

\def\mbb{\mathbb}
\def\mb{\mathbf}
\def\mc{\mathcal}
\def\R{\mbb{R}}
\def\Q{\mbb{Q}}
\def\Z{\mbb{Z}}
\def\C{\mbb{C}}
\def\N{\mbb{N}}
\def\F{\mbb{F}}
\def\K{\mathbb{K}}
\def\T{\mc{T}}
\def\A{\mc{A}}
\def\X{\mathfrak{X}}
\def\E{\exists}
\def\e{\epsilon}
\def\d{\delta}
\def\Ra{\Rightarrow}
\def\vs{\vspace{5mm}}
\def\La{\Leftarrow}
\def\and{\quad \text{and} \quad}
\def\rad{\mathrm{rad}}
\def\tt{\texttt}

\def\nemt{\neq \emptyset}
\newcommand{\dif}[2]{\frac{d#1}{d#2}}
\newcommand{\pardif}[2]{\frac{\partial #1}{\partial #2}}
\newcommand{\twopardif}[2]{\frac{\partial ^2 #1}{\partial #2^2}}
\newcommand{\Matrix}[1]{\begin{pmatrix} #1 \end{pmatrix}}
\newcommand{\Lint}[1]{\ointctrclockwise_{#1}}
\newcommand{\Ang}[1]{\left\langle #1 \right\rangle}
\newcommand{\Inn}{\langle \cdot , \cdot \rangle}
\newcommand{\norm}[1]{\left\| #1 \right\|}
\newcommand{\blanknorm}[1]{\| \cdot \|_{#1}}
\newcommand{\epfrac}[1]{\frac{\e}{#1}}
\newcommand{\ext}[1]{\tilde{\mb #1}}

\renewcommand{\epsilon}{\varepsilon}
\renewcommand{\bar}{\overline} 
\renewcommand{\hat}{\widehat}
%\renewcommand{\qedsymbol}{\FAX}


\declaretheoremstyle[
  headfont=\color{mypink}\normalfont\bfseries,
%  bodyfont=\color{red}\normalfont\itshape,
]{pink}

\declaretheoremstyle[
  headfont=\color{black}\normalfont\bfseries,
%  bodyfont=\color{red}\normalfont\itshape,
]{boxedsolution}

\theoremstyle{pink}
\newtheorem{definition}{Definition}

\theoremstyle{boxedsolution}
\newtheorem*{bproof}{Proof}
\newtheorem*{solution}{Solution}
  
\theoremstyle{definition}
\newtheorem{lemma}{Lemma}
\newtheorem{theorem}{Theorem}
\newtheorem{corollary}[definition]{Corollary}
\newtheorem{proposition}[definition]{Proposition}
\newtheorem{example}[definition]{Example}
\newtheorem{exercise}[definition]{Exercise}
\newtheorem{question}{Question}

\newtheoremstyle{claim}
{\topsep}{\topsep}{}{}%              
{\bfseries}{:}%             
{5pt plus 1pt minus 1pt}{}%             
\theoremstyle{claim}
% change this to \newtheorem*{claim}{Claim} to leave things un-numbered
\newtheorem{claim}{Claim}



\newenvironment{boxsol}
    {\begin{framed}
    \begin{solution}
    }
    {
    \end{solution}    
    \end{framed}}
    
\newenvironment{boxproof}
    {\begin{framed}
    \begin{bproof}
    }
    {
    \end{bproof} 
    \end{framed}}

\pagestyle{fancy}
\fancyhf{}
\lhead{MAT401 - Polynomial Equations and Fields}
\rhead{Nigel Petersen}
\cfoot{\thepage}

\begin{document}

There were a few common mistakes with these questions in particular, so I've drafted some more detailed sample solutions. Please let me know if you have any questions.

\vs

\setcounter{question}{1}

\begin{question}
Set $\alpha = \sqrt[3]{1 + \sqrt{3}}$. Determine $[\Q(\alpha); \Q(\sqrt{3})]$ and a basis for the extension.
\end{question}

\begin{proof}
Write $x = \alpha$ so that $x^3 = 1 + \sqrt{3}$, and hence $m_1(x) = x^3 - 1 - \sqrt{3} \in \Q(\sqrt{3})[x]$ is a monic polynomial containing $\alpha$ as a root. Note that by the monotonicity of $x^3$, we have that $\alpha$ is the only root of $m_1(x)$ over $\R$ and hence over $\Q(\sqrt{3})$. It will then suffice to show $\alpha \notin \Q(\sqrt{3})$ to show $m_1(x)$ is irreducible, and hence the minimal polynomial. Suppose towards a contradiction $\alpha \in \Q(\sqrt{3})$. Then $\Q(\sqrt{3})$ is a field containing $\Q$ and $\alpha$, so by minimality we have $\Q(\alpha) \subseteq \Q(\sqrt{3})$, namely $\Q(\alpha) \leq \Q(\sqrt{3})$ as $\Q$-vector spaces. However, we claim that $[\Q(\alpha) ; \Q] = 6$. Indeed, to see this, write $x = \alpha$ and hence $x^3 - 1 = \sqrt{3}$, so that $m_2(x) = x^6 - 2x^3 - 2 \in \Q[x]$ is a monic polynomial containing $\alpha$ as a root, and moreover $m_2(x)$ is irreducible over $\Q$ by Eisenstien with $p=2$. Thus, $m_2(x)$ is the minimal polynomial for $\alpha$ over $\Q$, and hence 
\[
[\Q(\alpha); \Q] = \deg(m_2(x)) = 6
\]
which is a contradiction as $[\Q(\sqrt{3}); \Q] = 2$. Thus, $\alpha \notin \Q(\sqrt{3})$ and hence $m_1(x)$ is the minimal polynomial for $\alpha$ over $\Q(\sqrt{3})$. Finally, we have that $[\Q(\alpha); \Q(\sqrt{3})] = \deg(m_2(x)) = 3$, and a basis is given by $\beta = \{1, \alpha, \alpha^2\}$.
\end{proof}

\vs

\begin{question}
    If $a,b \in \R$ for which $z = a+bi$ is algebraic over $\Q$, then $a$ and $b$ are algebraic over $\Q$.
\end{question}

\begin{proof}
    As $z$ is algebraic over $\Q$, there is a polynomial $q(x) \in \Q[x]$ for which $q(z) = 0$. Recall that we  have $q(\bar{z}) = 0$ by Assignment 3, and note that $p(x) = x^2+1 \in \Q[x]$ satisfies $p(i) = 0$. Thus each of $z, \bar{z}, i$ are algebraic over $\Q$, and hence over any extension of $\Q$. Finally, $E = \Q(z, \bar{z}, i)$ is a finite, and hene algebraic, extension of $\Q$ by the degree Lemma, as 
    \[
    [\Q(z, \bar{z}, i): \Q] \leq \deg(p(x))\deg(q(x))^2 < \infty
    \]
    Thus, $a$ and $b$ are algebraic over $\Q$ as $a = \frac{z + \bar{z}}{2} \in E$ and $b = \frac{x - \bar{z}}{2i} \in E$.
\end{proof}


\end{document}