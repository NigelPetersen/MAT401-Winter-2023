\documentclass[11pt]{article}
\usepackage[english]{babel}
\usepackage[utf8]{inputenc}
\usepackage[dvipsnames]{xcolor}
\usepackage[most]{tcolorbox}
\usepackage[linguistics]{forest}
\usepackage{fancyhdr}
\usepackage{amsmath}
\usepackage{amssymb}
\usepackage{amsfonts}
\usepackage{amstext}
\usepackage{amsmath,amssymb,amsthm, thmtools}
\usepackage{tikz,lipsum,lmodern}
\usepackage{array}
\usepackage{marvosym}
\usepackage{lastpage}
\usepackage{multicol}
\usepackage{tikz-cd}
\usepackage{array}
\usepackage{dirtytalk}
\usepackage{qtree}
\usepackage{framed}
\usepackage{enumitem}
\usepackage[hyperfootnotes=false]{hyperref}
\hypersetup{
colorlinks=true,
linkcolor={mypink}}

\definecolor{mypink}{RGB}{255, 50, 147}
\definecolor{pink}{RGB}{255, 0, 147}

\setlength{\textheight}{9.3in}
\setlength{\topmargin}{-0.7in}
\setlength{\textwidth}{6.5in}
\setlength{\oddsidemargin}{0in}
\setlength{\evensidemargin}{0in}
\setlength{\parskip}{6pt}
\setlength{\parindent}{0pt}

\def\mbb{\mathbb}
\def\mb{\mathbf}
\def\mc{\mathcal}
\def\R{\mbb{R}}
\def\Q{\mbb{Q}}
\def\Z{\mbb{Z}}
\def\C{\mbb{C}}
\def\N{\mbb{N}}
\def\F{\mbb{F}}
\def\K{\mathbb{K}}
\def\T{\mc{T}}
\def\A{\mc{A}}
\def\X{\mathfrak{X}}
\def\E{\exists}
\def\e{\epsilon}
\def\d{\delta}
\def\Ra{\Rightarrow}
\def\vs{\vspace{5mm}}
\def\La{\Leftarrow}
\def\and{\quad \text{and} \quad}
\def\rad{\mathrm{rad}}
\def\tt{\texttt}

\def\nemt{\neq \emptyset}
\newcommand{\dif}[2]{\frac{d#1}{d#2}}
\newcommand{\pardif}[2]{\frac{\partial #1}{\partial #2}}
\newcommand{\twopardif}[2]{\frac{\partial ^2 #1}{\partial #2^2}}
\newcommand{\Matrix}[1]{\begin{pmatrix} #1 \end{pmatrix}}
\newcommand{\Lint}[1]{\ointctrclockwise_{#1}}
\newcommand{\Ang}[1]{\left\langle #1 \right\rangle}
\newcommand{\Inn}{\langle \cdot , \cdot \rangle}
\newcommand{\norm}[1]{\left\| #1 \right\|}
\newcommand{\blanknorm}[1]{\| \cdot \|_{#1}}
\newcommand{\epfrac}[1]{\frac{\e}{#1}}
\newcommand{\ext}[1]{\tilde{\mb #1}}

\renewcommand{\epsilon}{\varepsilon}
\renewcommand{\bar}{\overline} 
\renewcommand{\hat}{\widehat}
%\renewcommand{\qedsymbol}{\FAX}


\declaretheoremstyle[
  headfont=\color{mypink}\normalfont\bfseries,
%  bodyfont=\color{red}\normalfont\itshape,
]{pink}

\declaretheoremstyle[
  headfont=\color{black}\normalfont\bfseries,
%  bodyfont=\color{red}\normalfont\itshape,
]{boxedsolution}

\theoremstyle{pink}
\newtheorem{definition}{Definition}

\theoremstyle{boxedsolution}
\newtheorem*{bproof}{Proof}
\newtheorem*{solution}{Solution}
  
\theoremstyle{definition}
\newtheorem{lemma}{Lemma}
\newtheorem{theorem}{Theorem}
\newtheorem{corollary}[definition]{Corollary}
\newtheorem{proposition}[definition]{Proposition}
\newtheorem{example}[definition]{Example}
\newtheorem{exercise}[definition]{Exercise}
\newtheorem{question}{Question}

\newtheoremstyle{claim}
{\topsep}{\topsep}{}{}%              
{\bfseries}{:}%             
{5pt plus 1pt minus 1pt}{}%             
\theoremstyle{claim}
% change this to \newtheorem*{claim}{Claim} to leave things un-numbered
\newtheorem{claim}{Claim}



\newenvironment{boxsol}
    {\begin{framed}
    \begin{solution}
    }
    {
    \end{solution}    
    \end{framed}}
    
\newenvironment{boxproof}
    {\begin{framed}
    \begin{bproof}
    }
    {
    \end{bproof} 
    \end{framed}}

\pagestyle{fancy}
\fancyhf{}
\lhead{MAT401 - Polynomial Equations and Fields}
\rhead{Nigel Petersen}
\cfoot{\thepage}

\begin{document}

\section{Finding inverses in simple extensions}

For the time being, we will assume that $F$ is a field, $q(x) \in F[x]$ is a monic, irreducible polynomial over $F$, $E$ is an extension of $F$, and $\alpha \in E$ is a root of $q(x)$. By this construction, $\alpha \in E$ is algebraic over $F$, and so $F(\alpha) \cong F[x] / \Ang{q(x)}$, and moreover a basis for $F(\alpha)$ is $\beta = \{1, \alpha, \dots, \alpha^{m-1}\}$, where $m=\deg(q)$. Consider the isomorphism $\psi: F(\alpha) \to F[x]/\Ang{q(x)}$ such that $\psi(1) = [1]$ and $\psi(\alpha) = [x]$. To find the inverse of a non-zero element $\gamma \in F(\alpha)$, we proceed as follows:
\begin{enumerate}
    \item Write $\psi(\gamma) = [p(x)]$ for some polynomial of degree at most $m-1$, where $\psi(\gamma) \neq 0$ as $\gamma \neq 0$ and $\psi$ is injective.

    \item Using the division algorithm in $F[x]$, we can compute $s_1, s_2 \in F[x]$ for which 
    \[
    1 = q(x)s_1(x) + p(x)s_2(x)
    \]

    \item Pushing the above into $F[x]/\Ang{q(x)}$ using the canonical projection map $\pi(x) = [x]$, we can thus write $[1] = [q(x)s_1(x) + p(x)s_2(x)] = [s_2(x)][p(x)]$. This constructs our candidate for $\psi(\gamma)^{-1} \in F[x]/\Ang{q(x)}$. 

    \item Lastly, we can pull back into $F(\alpha)$ using $\psi^{-1}$ (as $\psi^{-1}([x]) = \alpha$) to obtain $\gamma^{-1} = \psi^{-1}([s_2(x)])$.
\end{enumerate}

\vs

\begin{example}
Consider $q(x) = x^3+9x+6 \in \Q[x]$. Prove that $q(x)$ is irreducible, and that $\E \alpha \in \R$ such that $q(\alpha) = 0$. Compute the inverse of $1 + \alpha \in \Q(\alpha)$.
\end{example}

\begin{proof}[Solution]
Note that $q(x)$ is an odd degree polynomial over $\Q$, and hence $\R$, so by the Intermediate Value Theorem, $q$ has a root $\alpha \in \R$. However, if $\alpha \in \Q$, by the Rational Root Theorem, we must have that $\alpha = \pm 6$, where $q(\pm 6) \neq 0$, and hence $\alpha \in \R \setminus \Q$.
Using the map $\psi$ as defined in the algorithm above, we can write $\psi(\gamma):=\psi(1 + \alpha) = [1+x] \in \Q[x]/ \Ang{q(x)}$, where $p(x) = 1+x$ as in the algorithm. Using long division, we can write 
\[
x^3+9x+6 = (x+1)(x^2-x+10) - 4 \quad \iff \quad 1 = -\frac{1}{4}(x^3+9x+6) + \frac{1}{4}(x^2-x+10)(x+1)
\]
Thus, we have $[s_2(x)] := \left[\frac{1}{4}(x^2-x+10)\right]$ as the inverse of $[p(x)] = [1+x] = \psi(\gamma) \in \Q[x]/\Ang{q(x)}$, and hence our desired inverse is $(1+\alpha)^{-1}=\psi^{-1}([s_2(x)])$, where 
\[
\psi^{-1}([s_2(x)]) = \frac{1}{4}\alpha^2 - \frac{1}{4}\alpha + \frac{10}{4}
\]
For our sanity, we can check that what we have is truly an inverse. Indeed, we have
\begin{align*}
    (1 + \alpha)\left( \frac{1}{4}\alpha^2 - \frac{1}{4}\alpha + \frac{10}{4}\right) &= \frac{1}{4}\alpha^2 - \frac{1}{4}\alpha + \frac{10}{4} + \frac{1}{4}\alpha^3 - \frac{1}{4}\alpha^2 + \frac{10}{4}\alpha 
    \\&= \frac{1}{4}\alpha^3 + \frac{9}{4}\alpha + \frac{10}{4}
    \\&= \frac{1}{4}(q(\alpha) + 4)
    \\&= 1 \qedhere
\end{align*}
\end{proof}

% \newpage

% \begin{definition}
%     Let $F$ be a field, $K$ an extension of $F$ and $\alpha \in K$ algebraic over $F$. We define a minimal polynomial of $\alpha$ over $F$ as a monic, irreducible polynomial $m(x) \in F[x]$ for which $m(\alpha) = 0$.
% \end{definition}

% It's worth noting that this is not the true definition, but an equivalent one (the true definition requires a bit of set up). The course notes hint at the idea of the minimal polynomial (I will refer to it as \textit{the} minimal polynomial as it is in fact unique), and so some of its properties you are already familiar with. In particular, the following result (essentially Exercise 8.5 and Exercise 9.1)

% \begin{proposition}
%     Let $F$ be a field, $K$ an extension of $F$ and $\alpha \in K$ algebraic over $F$. Then
%     \[
%     F[\alpha] \cong F[x] / \Ang{m(x)}
%     \]
%     where $m(x)$ is the minimal polynomial of $\alpha$ over $F$. Moreover, as $F[x]/\Ang{m(x)}$ is a field, we have $F(\alpha) = F[\alpha]$, and 
%     $[F(\alpha):F] = \deg(m(x))$ with basis $\{1, \alpha, \dots, \alpha^{n-1}\}$, where $\deg(m(x)) = n$.
% \end{proposition}

% There were a few common mistakes with these questions in particular, so I've drafted some more detailed sample solutions. Please let me know if you have any questions.

% \vs

% \setcounter{question}{1}

% \begin{question}
% Set $\alpha = \sqrt[3]{1 + \sqrt{3}}$. Determine $[\Q(\alpha); \Q(\sqrt{3})]$ and a basis for the extension.
% \end{question}

% \begin{proof}
% Write $x = \alpha$ so that $x^3 = 1 + \sqrt{3}$, and hence $m_1(x) = x^3 - 1 - \sqrt{3} \in \Q(\sqrt{3})[x]$ is a monic polynomial containing $\alpha$ as a root. Note that by the monotonicity of $x^3$, we have that $\alpha$ is the only root of $m_1(x)$ over $\R$ and hence over $\Q(\sqrt{3})$. It will then suffice to show $\alpha \notin \Q(\sqrt{3})$ to show $m_1(x)$ is irreducible, and hence the minimal polynomial. Suppose towards a contradiction $\alpha \in \Q(\sqrt{3})$. Then $\Q(\sqrt{3})$ is a field containing $\Q$ and $\alpha$, so by minimality we have $\Q(\alpha) \subseteq \Q(\sqrt{3})$, namely $\Q(\alpha) \leq \Q(\sqrt{3})$ as $\Q$-vector spaces. However, we claim that $[\Q(\alpha) ; \Q] = 6$. Indeed, to see this, write $x = \alpha$ and hence $x^3 - 1 = \sqrt{3}$, so that $m_2(x) = x^6 - 2x^3 - 2 \in \Q[x]$ is a monic polynomial containing $\alpha$ as a root, and moreover $m_2(x)$ is irreducible over $\Q$ by Eisenstien with $p=2$. Thus, $m_2(x)$ is the minimal polynomial for $\alpha$ over $\Q$, and hence 
% \[
% [\Q(\alpha); \Q] = \deg(m_2(x)) = 6
% \]
% which is a contradiction as $[\Q(\sqrt{3}); \Q] = 2$. Thus, $\alpha \notin \Q(\sqrt{3})$ and hence $m_1(x)$ is the minimal polynomial for $\alpha$ over $\Q(\sqrt{3})$. Finally, we have that $[\Q(\alpha); \Q(\sqrt{3})] = \deg(m_2(x)) = 3$, and a basis is given by $\beta = \{1, \alpha, \alpha^2\}$.
% \end{proof}

% \vs

% \begin{question}
%     If $a,b \in \R$ for which $z = a+bi$ is algebraic over $\Q$, then $a$ and $b$ are algebraic over $\Q$.
% \end{question}

% \begin{proof}
%     As $z$ is algebraic over $\Q$, there is a polynomial $q(x) \in \Q[x]$ for which $q(z) = 0$. Recall that we  have $q(\bar{z}) = 0$ by Assignment 3, and note that $p(x) = x^2+1 \in \Q[x]$ satisfies $p(i) = 0$. Thus each of $z, \bar{z}, i$ are algebraic over $\Q$, and hence over any extension of $\Q$. Finally, $E = \Q(z, \bar{z}, i)$ is a finite, and hene algebraic, extension of $\Q$ by the degree Lemma, as 
%     \[
%     [\Q(z, \bar{z}, i): \Q] \leq \deg(p(x))\deg(q(x))^2 < \infty
%     \]
%     Thus, $a$ and $b$ are algebraic over $\Q$ as $a = \frac{z + \bar{z}}{2} \in E$ and $b = \frac{x - \bar{z}}{2i} \in E$.
% \end{proof}


\end{document}